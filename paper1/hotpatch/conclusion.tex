\section{Conclusion}
\label{sec:conclusion}

We introduce a method for hot patching: a technique we believe to be a
promising alternative to redundancy, ad hoc self--healing techniques,
``patch and pray,'' or other approaches to dynamic software updates.
Hot patching has the potential for aligning actual practices with
acknowledged ``best practices'' relating to critical security or 
functionality updates.  We hold that one major impediment to hot
patching is the opaque nature of most patches (be it
proprietary or open software), and our method of patching along with the
PO file format are first attempts at providing a basis for informed reasoning
about the structure and implications of a patch. 

We present a reasoned approach to making patching a part of the
standard tool chain. We demonstrate a working binary userland patcher
operating completely at the object level. Our system is, to our
knowledge, the first to utilize DWARF type information to automate the
transformation between old and new versions of a type. There yet
remains much work to be done, and our future work involves support for
patching multi-threaded targets, better support for handling opaque
types such as \texttt{void}*, and further development of patch
versioning and the ability to perform operations on patch objects.
